\chapter{Einf\"uhrung K\"unstliche Intelligenz}

Damit ein Fahrzeug autonom, d.\,h. ohne menschliches Zutun, im
Strassenverkehr reagieren kann, muss es Faehig sein, die verschiedenen
Verkehrsteilnehmer zu identifizieren. Hierbei wird sich einem
Teilgebiet der angewandten Informatik bedient - der \ac{KI}, auch
\ac{AI} genannt.

%\noindent\textbf{Overview}
\section*{Overview}

Hier werden folgende Fragen beantwortet:

\begin{itemize}
  \item Was ist Artificial Intelligence
  \item Modellbildung Neueronaler Netze
  \item Technische Umsetzung
  \item Uebersicht: Convolutional Neueral Networks
  \item Training eines Neuronalen Netzes
  \item Der YOLO Algorithmus
  \item Bisheriger Stand im Projekt - Status Quo
\end{itemize}

%%\noindent\textbf{Was ist Kuenstliche Intelligenz}
\section*{Was ist K\"unstliche Intelligenz}

\begin{center}\emph{
  Artificial Intelligence is the study of how to make computers do things at which, at the mo-
  ment, people are better.\footnote{\cite{bosl}}
}\end{center}

\ac{KI} ist eine Teildisziplin der angewandten Informatik, in der es
um automatisierte Problemloesungen geht. Die Urspruenge lagen im
Ansatz, allein Logik zu verwenden. Nachteil dabei: hohe Komplexitaet
fuer Abdeckung aller Faelle. Es wurde als neuer Ansatz das menschliche
Gehirn als Grundlage verwendet. Menschen reagieren auf eine sich
veraendernde Umwelt. Wir lernen aus Beispielen. Es erfolgt keine
Abarbeitung von Schlussregeln wie ``WENN \dots , DANN \dots''. Man
stelle sich das Halten des Gleichgewichts beim Fahrradfahren oder das
Schreiben auf einem Blatt Papier vor. Die Reaktionszeiten fuer eine
sequenzielle Abarbeitung waere viel zu hoch. Beim Fahrradbeispiel
waere man schon laengst vor der Abarbeitung aller Regeln
umgeflogen. Das selbe gilt analog fuer den Fall des Schreibens. Es
wuerde einfach viel zu lange dauern und die Regeln waeren viel zu
komplex.

Unser Gehirn funktioniert anders. Es passt sich den Situationen an und
greift dabei auf Erfahrungen von schon erlebten Situationen
zurueck. Das wollte man auch fuer Maschinen -- dynamische Anpassung
auf sich aendernde Randbedingungen. Das ist auch der Grund, warum
kuenstliche Netze antrainiert werden muessen. Sie sollen auf
Situationen reagieren und dabei auf `Erfahrung von schon erlebten`
zurueckgreifen. Das `Schon Erlebte` stellt dabei das Trainingsmaterial
dar, mit dem das Netzwerk antrainiert wird. Dennoch, im Allgemeinen
ist zu entscheiden:

\begin{itemize}
  \item Existiert ein Algorithmus, dann ist dieser vorzuziehen.
  \item Ist Erfahrung vorhanden, ein Problem anhand von Regeln zu
    beschreiben, ist dies vorzuziehen.
  \item Wenn die ersten beiden Ansaetze nicht zum Erfolg fuehren oder
    nicht fuehrten, dann kann der Einsatz von \ac{KI} zielfuehrend
    sein, wenn genuegend Daten vorhanden sind, aus denen gelernt
    werden kann.
\end{itemize}

Im Strassenverkehr aendern sich zu jedem Zeitpunkt die
Randbedingungen. Autos biegen ab, bremsen, ueberholen. Es gibt
Verkehrsschilder, Ampelanlagen, Fussgaenger- und Fahrradwege und
mehr. Menschen greifen im Strassenverkehr auf ihre Erfahrung
zurueck. Moechte man als Ziel ansetzen, dass Fahrzeuge in der Lage
sein sollen, im Verkehr ohne menschliches Zutun sich zu bewegen, ist
eines ersichtlich. Die \ac{KI} ist prädestiniert fuer den Einsatz im
Bereich \emph{autonomes Fahren}.

%% Befehl, dass dennoch voll ausgeschrieben - zwingen
%%\noindent\textbf{\ac{KNN}}\\
\section*{\ac{KNN}}

%% Aufbau/Modell
\acp{KNN} versuchen das menschliche Gehirn nachzubilden. Wie unsere
Nervenzellen trainierbar sind und Steuerungsaufgaben uebernehmen, so
moechte man auch, dass Computerprogramme aehnlich befaehigt sein
sollen, in analoger Weisse auf neue Situationen aus schon erlebten
Beispielen zu reagieren. Man bediehnt sich hierbei dem Gehirn als
biologischen Bauplan \autoref{fig:neuron}.

\begin{figure}[!htb]
  \includegraphics[width=1\textwidth]{obj-rec/neuron.png}
  \caption{Menschliches Neuron; \cite{bosl}}
  \label{fig:neuron}
\end{figure}

Die Dendriten nehmen das Signal auf, leiten es an den Zellkoerper
weiter. Dort findet eine Gewichtung der Informationen statt. Die
gewichtete Gesamtinformation geht ueber das Axon weiter und verteilt
sich auf die Synapsen, welche jeweils mit weiteren Dendriten anderer
Neuronen verbunden sind. Ein einzelnes Neuron wird in Form eines
Softwarebausteins nachgebildet. Ein Neuron \emph{j} besteht aus:

\begin{itemize}
  \item $k$ gewichteten Eingaengen $W_{kj}$
  \item aus der Summe $net_j$ der jeweils einzelnen Produkten der
    Gewichte $W_{kj}$ und dem Wert des Eingangs
    $O_k$. $net_j$ stellt damit die Information zusammen,
    die aus dem Netz in das Neuron eingehen.
  \item der Aktivitaet des Neurons, d.\,h. wie stark der potenzielle
    Einfluss von $net_j$ auf andere Neuronen sein kann. $\Theta_j$ ist
    ein \emph{Hyper Parameter}.
  \item der Ausgabe -- Verbindung zu anderen Neuronen
\end{itemize}

\autoref{fig:art-neuron} entspricht der Beschreibung.

\begin{figure}[!htb]
  \centering
  \includegraphics[width=0.8\linewidth]{obj-rec/artificial-neuron.png}
  \caption{Modellierung eines einzelnen Neurons -- Kuenstliches Neuron; \cite{bosl}}
  \label{fig:art-neuron}
\end{figure}

Viele Neuronen sind so miteinander verbunden und bilden zusammen ein
\emph{Neuronales Netzwerk}. Als technische Modellierung existieren
verschiedene Varianten, von denen aber hier nur die \emph{forwaets
gerichtete} Variante interessiert.

\begin{figure}[!htb]
  \centering
  \subfloat[][]{\includegraphics[width=0.25\linewidth]{obj-rec/forward-looking-network-v1}}%
  \qquad
  \subfloat[][]{\includegraphics[width=0.25\linewidth]{obj-rec/forward-looking-network-v2}}%
  \qquad
  \subfloat[][]{\includegraphics[width=0.25\linewidth]{obj-rec/forward-looking-network-v3}}%
  \caption{Forward Looking Networks}
  \label{fig:for-lok-net}
\end{figure}

\autoref{fig:for-lok-net} zeigt verschiedene Ausfuehrungen eines
\emph{Feed Forward Neural Networks} oder auch \emph{Fully Connected
Neural Network}. Das entspricht der klassischen Vorstellung eines
\acp{KNN}. Jedes Neuron ist mit jedem vorherigen und jedem
nachfolgenden Neuron verbunden.

Die erste Schicht bildet den Eingang. Dort treffen Signale ein,
z.\,B. in Form eines Bildes. Es geht weiter in die zweite
Schicht. Tiefer liegende Schichten werden auch als \emph{hidden layer}
bezeichnet, da sie von aussen, d.\,h von den Schnittstellen, nicht zu
sehen sind. Die letzte Schicht ist die Schnittstelle nach
Aussen. Jedes Neuron gibt eine Wahrscheinlichkeit ueber den
representierenden \emph{Classifier} aus. Zum Beispiel als
Anwendungsfall in der Objekterkennung, ob Objekt von der Klasse
`Afrikanischer Elefant` ist. Existieren mehrere Neuronen in der
letzten Schicht, kann das Netz Aussagen ueber mehrere Classifier
machen.

Es gibt noch wesentlich mehr Architekturen/Topologien, wie ein
\ac{KNN} aufgebaut sein kann. Jede hat ihre Vor- und Nachteile und
somit eigene Anwendungsgebiete. In der Objekterkennung,
Bildverarbeitung wird eine andere Variante als die fully connected
Variante eingesetzt - das \ac{CNN}. Eine nicht vollstaendige Liste ueber
verschiedene Topologien / Architekturen von Kuenstliche Neuronalen
Netzwerken ist in \autoref{fig:topo-ann} zu sehen.

\begin{figure}[!htb]
\includegraphics[width=1\linewidth]{obj-rec/topo-ann.png}
  \caption{Eine nicht vollstaendige grobe Uebersicht ueber Topologien
    / Architekturen von Kuentlich Neuronalen Netzwerken}
  \label{fig:topo-ann}
\end{figure}

\section*{Uebersicht CNN}

Im Gegensatz zum \emph{Fully Connected Network} sind die Neuronen des
\acp{CNN} nicht mit jedem Neuron der uebergeordneten Schicht
verbunden. Die Verbindung entsteht als Faltungsoperation mit einem
Filter bestimmter Groesse (Kernel-Size). Der Filter `wandert` ueber das
Bild und verknuepft so jeden Bildbereich mit dem darueber liegenden
Neuronen. Jedes Neuron ist somit mit einem Filter verknuepft. Die
Ergebnisse der Faltung entsprechen den Gewichten, mit denen das Neuron mit der
darunter liegende Schicht verbunden ist. Allgemein gibt es mehrere
Filter pro Schicht; dementprechend besteht jede Schicht aus mehreren
Neuronen. In \autoref{fig:conv-schema} ist das Vorgehen aufskizziert.

\begin{figure}[!htb]
\centering\includegraphics{obj-rec/conv-schema.png}
  \caption{ Jedes Neuron ist mit einer Faltungsoperation (Filter) mit
    dem darueber liegenden Neuron verbunden. Vorteil eines \acp{CNN}
    ist: gelernte Objekte muessen nur einmal gelernt werden, egal wo
    im Bild zu sehen (ob oben oder unten, links oder rechts). Das gilt
    nicht fuer \emph{Fully Connected Networks}. Dort muss ein und das
    selbe Objekt mehrmals gelernt werden, wenn im Bild wo anders
    liegt.}
  \label{fig:conv-schema}
\end{figure}

Die Parameter der Filter haben anfangs zufaellige Werte. Die Werte
werden im Trainingsprozess `gelernt` oder `antrainiert`. Die Faltung,
d.\,h. der Filter, wird nicht nur auf den Input (das Bild) , sondern
auch zur Reduktion \emph{Feature Extraction} der inneren Schichten
\emph{Hidden Layers} angewandt. Man kann fuer die inneren Schichten ein
schon vortrainiertes, voellig beliebiges \acp{CNN} verwenden. Vorteil
ist, es muss weniger Zeit fuer das Lernen seines Anwendungsgebietes
aufgewendet werden und die \emph{Feature Extraction} ist schon
aufgebaut. Es gibt hierbei zwei wesentliche Punkte zu beachten:

\begin{itemize}
  \item Es muss `nur` der Eingang auf die korrekte Groesse des
    Bildes angepasst werden.
  \item Die letzte Schicht muss ein \emph{YOLO-Layer} sein mit den
    gewuenschten Ausgangsneuronen fuer jede Klasse an zu
    detektierenden Objekten.
\end{itemize}

\begin{figure}[!htb]
\includegraphics[width=1\linewidth]{obj-rec/yolo-v2-schema.png}
  \caption{
    Links (hellblau) der Eingang (der Situation angepasst), in der Mitte (dunkelblau)
    das pretrained CNN Network, rechts (braun) das YOLO-Layer (auf
    Beduerfnisse angepasst)}
  \label{fig:yolo-v2-schema}
\end{figure}

In \autoref{fig:yolo-v2-schema} ist die Struktur des YOLO-Netzwerks
aufskizziert. Es gibt noch sehr viel mehr zu beachten und Parameter
einzustellen. 

%%\noindent\textbf{Training eines \ac{KNN}}
\section*{Training eines Kuenstlich Neuronalen Netzes}


\section*{YOLO Algorithmus}
