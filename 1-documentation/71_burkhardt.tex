\chapter{Anwendung der Theorie auf Praxis}

In der Objekterkennung hat sich der YOLO Algorithmus bewaehrt. Die
YOLO Variante untescheidet nur in den Letzten Schichten. Ansonsten
sind CNN und YOLO von der Architektur identisch.

Es gibt mehrere Technologien, die den Aufbau eines Neuronalen
Netzwerks unterstuetzen. Einige Bibliotheken in Python sind
beispielsweise: PyTorch, Keras, Tensor Flow. Auch Matworks stellt ein
eigenes Framwork bereit und ist in der Matlab eigenen Sprache zu
bedienen. Vorteil von Matlab ist: Es gibt sehr viele speziell auf
Matlab angepasste Tools fuer den kompletten Erstellungs-, Evaluierung-
und Anwendungsprozesses. Hingegen die frei zugaenglichen Alternativen
bieten einen kleineren Anwendungsbereich. Beispielweisse, wenn fuer
die Evaluierung im Bild die Objekte gekennzeichnet werden sollen,
muessen andere Tool als z.\,B. nur Tensor Flow verwendet werden wie
OpenCV zur Bildverarbeitung. OpenCV reichert das Bild mit Rahmen um
erkannte Objekte an und setzt Labels. In Matlab ist alles dabei und
man externen Tools sind unnoetig.

Hier im Projekt wird Matlab als Entwicklungsumgebung verwendet, weil
Vorgaengerarbeiten darauf aufgebaut haben. Wuerde man sich dagegen
Entscheiden, muesste der Erstellungsprozess von vorne begonnen werden.

%\noindent\textbf{Overview}\\
\section*{Overview}

Hier werden folgende Fragen beantwortet:

\begin{itemize}
  \item Erstellen, Trainiern, Testen
  \item Evaluierung mit schon vorhandenem Netzwerk
  \item Deployment auf Jetson Nano
\end{itemize}

\section*{Erstellung YOLO-KNN}

\section*{Evaluierung - Vergleich Mit Vorgaengernetz}

\section*{Deployment Auf Jetson Nano}

Schritte, die Noetig sind:

\begin{itemize}
  \item YOLO Netzwerk
  \item Main-Funktion erstellen
\end{itemize}

\textbf{YOLO Netzwerk}: CNN, das \dots

\textbf{Main-Funktion}: Funktion, die in einer Endlosschleife Bilder von
der WebCam aufniehmt und dem Detektor uebergibt. Der Detektor scannt
das aufgenommene Bild nach Objekten. Wenn Objekte gefunden wurde, dann
wird die entsprechende Bounding Box und Label auf das Bild gebunden
und anschliessend auf dem Bildschirm angezeigt. Hierbei muss entweder
ein externer Monitor an den Jetson angeschlossen werden oder eine
Remoteverbindung hergestellt werden.
