\chapter{Einleitung}\label{Kap:Einleitung}
Die vorliegende Arbeit entstand im Rahmen eines Mechatronischen Projekts des Studiengangs Systems Engineering an der Hochschule Aalen. Hierdurch konnte in Form eines anwendungsorientierten Projekt die praktischen Umsetzung der erlangten theoretischen Kenntnisse ermöglicht werden.\\
Durch die zunehmende Digitalisierung und technische Fortschritte im Bereich des maschinellen Lernens finden in der praktischen Anwendung durch die Automobilindustrie immer weitgehendere Assistenzsysteme Anwendung. Diese Fortschritte stellen schlussendlich ein Schlüsselelement hin zum automatisierten oder gar autonomen Fahren dar. Aufgrund dieses aktuellen Anlasses sollte  die Grundlage geschaffen werden, diese Technologien auf ein ferngesteuertes Fahrzeug unter Verwendung eines Fahrbahnmodells anwenden zu können.\\
Durch vorliegende Arbeit sollte hierbei ein Kollisionsvermeidungsassistent auf der Basis eines Neuronalen Netzwerkes mittels einer YOLOv2-Architektur geschaffen werden, um vorausfahrende Objekte erkennen und nachfolgend entsprechend reagieren zu können. Dies bedingte unter Anderem folgende Aufgabenstellungen:
\begin{itemize}
	\item Beschaffung geeigneter Trainings- und Evaluierungsdaten
	\item Erstellung eines Frameworks zum:
	\begin{itemize}
		\item Erstellen eines KNN
		\item Aufbereiten der Datensätze
		\item Trainieren eines KNN
		\item Evaluieren eines KNN
	\end{itemize}
	\item Deployment auf eine Target-Hardware mit:
	\begin{itemize}
		\item Erstellung der notwendigen Funktionen
		\item Implementierung der für das Targer notwendigen Software
		\item Benchmarking verschiedener KNN
	\end{itemize}
	\item Abschließende Empfehlung zur praktischen Anwendung und Fortführung 
\end{itemize}     
Diese jeweiligen Aufgabenstellungen sollen nun nachfolgend genauer beschrieben und deren Lösung genauer erläutert werden.