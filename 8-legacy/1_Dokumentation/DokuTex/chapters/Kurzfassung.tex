%===============================================================================
% Kurzfassung
%
% 29.05.2020  Stefan Briem
% ===============================================================================
%-------------------------------------------------------------------------------
\section*{Kurzfassung}

Durch eine zunehmende Digitalisierung und Fortschritte im Bereich des Maschinellen Lernens ergeben sich umfassende Anwendungsmöglichkeiten dieser Technologien. Diese können insbesondere auch zur Weiterentwicklung von Fahrzeugen hin zu autonomem Verkehr genutzt werden. \newline
Ziel der vorliegenden Arbeit war es hierbei, diese wissenschaftlichen Erkenntnisse auf ein ferngesteuertes Modellfahrzeug und einen Fahrbahnsimulator zu übertragen.\newline
Dies beinhaltete insbesondere die Schaffung eines Framework zum Erstellen, Trainieren, und Evaluieren eines Neuronalen Netzwerks zur Fahrzeugerkennung auf Basis einer YOLOv2-Architektur. Auf dieser Basis wurde ein zweites Framework zum Deployment eines Detektionsalgorithmus auf die Target-Platform in Form eines Raspberry Pi erstellt, mithilfe dessen unterschiedliche Konfigurationen der Netzwerke auf unterschiedlichen Targets einem Benchmarking unterzogen werden konnten. Die Ergebnisse dieser Evaluierung bildeten im Folgenden die Grundlage einer Umsetzungsempfehlung für weitergehende Anwendungen, sodass die im Zuge vorliegender Arbeit erlangten Erkenntnisse und die Funktionen der Frameworks hierfür als Basis verwendet werden können.
%-------------------------------------------------------------------------------


%-------------------------------------------------------------------------------
\vfill
%-------------------------------------------------------------------------------
\section*{Abstract}
Increasing digitalisation and advances in machine learning are opening up a wide range of applications for these technologies. These can be used in particular for the further development of vehicles towards autonomous traffic.\newline
The aim of this thesis was to transfer these scientific findings to a remote-controlled model vehicle and a road simulator.\newline
This included in particular the creation of a framework for the creation, training, and evaluation of a neural network for vehicle recognition based on a YOLOv2 architecture. On this basis, a second framework for the deployment of a detection algorithm on the target platform of a Raspberry Pi was created. This could be used to benchmark different configurations of the networks on different targets. The results of this evaluation formed the basis for a recommendation for further applications. Therefore the knowledge, which was gained in the course of the present work and the functions of the frameworks can be used as a basis for these applications.

% EOF