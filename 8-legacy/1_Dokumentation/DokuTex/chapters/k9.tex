\chapter{Ausblick}\label{Kap:Ausblick}
%08.07.2020
%===============================================================================

Bei einer praktischen Weiterführung des Projektes zur Implementierung eines Kollisionsvermeidungsassistenten können die im Zuge der vorliegenden Arbeit geschaffenen Grundlagen folgendermaßen genutzt und erweitert werden:\newline
Um die Detektionsrate weiter steigern zu können wird ein weitaus umfangreicherer Datensatz zum Training des Netzwerkes benötigt. Dies trifft insbesondere dann zu, wenn nicht nur eine Target-Klasse, wie hier beispielsweise Autos, sondern mehrere Klassen, wie zusätzlich Fußgänger, Radfahrer oder andere Straßenverkehrsteilnehmer, detektiert werden sollen. Diese Forderung kann vor Allem durch die Verwendung des während der Arbeit erstellten Vidoe-Frameworks unter Zuhilfenahme des Video-Labelers erfüllt werden, da hierdurch sehr umfangreiche Datensätze generiert werden können. Durch die zusätzliche Anwendung von Data-Augmentation kann dieser Umfang nochmals erhöht und gleichzeitig einem Overfitting des Netzwerkes vorgebeugt werden. 
\newline
Diese Grundlagen können anschließend auf den Fahrzeugsimulator übertragen werden, wobei durch eine Verknüpfung von auf dem Bildschirm angezeigten Videosequenzen mit einer hierzu passenden Fahrbahn-Trajektorie ein sehr realistisches Ergebnis erzielt werden kann. Hierdurch kann anhand einer realistischen Anwendung das große Innovationspotential aufgezeigt werden, welches die Anwendung Maschinellen Lernens in der Industrie bereitstellt.


%===============================================================================
% EOF