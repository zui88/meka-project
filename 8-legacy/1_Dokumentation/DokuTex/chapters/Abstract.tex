\section*{Abstract}

The capture of the relative position is made, on demand of the AGCO GmbH, only visually by a time-of-flight camera system that will be installed on the SEM. In this thesis, an appropriate strategy for the identification of the trailer is acquired by means of the delivered raw data of the camera system in consideration of the characteristic properties of the camera and the described application. In order to determine the target adjustment, kinematic models of the ejection mechanism and dynamic models of the flight path for the crop are developed. Based on this, appropriate adjustment strategies are designed. A position control is delineated to realize the target adjustment in consideration of the system properties of the used test vehicles, especially the ejection mechanism.\\
The acquired approaches are developed model-based. Therefore, they were first implemented in function algorithms and tested by simulation. Then the algorithms are integrated in the test vehicle by a rapid control prototyping system and the necessary interfaces are added to the hardware and software of the vehicle. The verification of the acquired approaches is carried out by the test vehicles on the test track by means of appropriate scenarios and assessment criteria. \\
The developed capture algorithms can prove robust and adequate detection of the relative position between SEM and trailer on straight-line drives, in curves and the transition between straight and curved driving elements. The developed control shows a good command action and the approach for the determination of the target adjustment of the ejection mechanism provides qualitatively plausible results. For this reason, the developed concept for the controlled adjustment of the ejection mechanism is evaluated as functioning.
