\addsec{\label{sec:abstract}Zusammenfassung / Abstract}

\minisec{Abstract}

In dieser Arbeit wird mit einem K\"unstlich Neuronalen Netzwerk (KNN) ein
Objekterkennungsalgorithmus mit Matlab entworfen, der in Echtzeit Fahrzeuge
erkennt. Der Algorithmus wird auf einen Jetson Nano von Nvidia installiert und
dort auf der Graphik Prozessor Unit (GPU) ausgef\"uhrt. Werden Objekte erkannt,
wird die Information einer Verarbeitungspipeline \"ubergeben. Die Pipeline
manipuliert die gewonnenen Daten in gew\"unschter Wei{\ss}e. Am Ende sollen die
Informationen \"uber die Objekte \"uber das Control Area Network (CAN) versendet
werden. Da der Jetson Nano \emph{nicht} \"uber einen eigenen CAN Controller
verf\"ugt, wird eine M\"oglichkeit besprochen, wie die Kommunikation
\"uber den Bus realisiert werden kann.\\

\minisec{\label{abstract}Zusammenfassung}

Im ersten Teil wird besprochen, was ein K\"unstlich Neuronales Netwerk ist und
warum es sich f\"ur die Objekterkennung eignet. Im zweiten Teil wird der Jetson
Nano vorgestellt, will hei{\ss}en die Hardware (Developement-Board, Platine). Im
dritten Teil wird die Erstellung und die Installation des
Objekterkennungsalgorithmus vorgestellt. Im vierten Teil wird die
Verarbeitungspipeline besprochen, was diese ist und wie sie sich ins Projekt
einf\"ugt. Im f\"unften und letzten Teil werden die einzelnen Schritte der
Arbeit nochmals in einem gr\"o{\ss}eren Zusammenhang betrachten. Es wird die
schlu{\ss}endliche Versendung der Nachrichten via CAN erl\"autert. Weiterhin
wird ein Ausblick gegeben, wie das Projekt weiter gef\"uhrt werden kann und
welche Punkte im Laufe der Arbeit auftauchten, die verbessert werden k\"onnen.
