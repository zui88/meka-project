\addsec{\label{sec:abstract}Zusammenfassung / Abstract}


\minisec{\label{zusammenfassung}Zusammenfassung}

In dieser Arbeit wird mit einem Künstlich Neuronalen Netzwerk (KNN) ein
Objekterkennungsalgorithmus mit Matlab entworfen, der in Echtzeit Fahrzeuge
erkennt. Der Algorithmus wird auf einen Jetson Nano von Nvidia installiert und
dort auf der Graphik Prozessor Unit (GPU) ausgeführt. Werden Objekte erkannt,
wird die Information einer Verarbeitungspipeline übergeben. Die Pipeline
manipuliert die gewonnenen Daten in gewünschter Weiße. Am Ende sollen die
Informationen über die Objekte über das Control Area Network (CAN) versendet
werden. Da der Jetson Nano nicht über einen eigenen CAN Controller verfügt, wird
eine Möglichkeit besprochen, wie die Kommunikation über den Bus realisiert
werden kann.\\

Im ersten Teil wird besprochen, was ein Künstlich Neuronales Netwerk ist und
warum es sich für die Objekterkennung eignet. Im zweiten Teil wird der Jetson
Nano vorgestellt, will heißen die Hardware (Developement-Board, Platine). Im
dritten Teil wird die Erstellung und die Installation des
Objekterkennungsalgorithmus vorgestellt. Im vierten Teil wird die
Verarbeitungspipeline besprochen, was diese ist und wie sie sich ins Projekt
einfügt. Im fünften und letzten Teil werden die einzelnen Schritte der Arbeit
nochmals in einem größeren Zusammenhang betrachten. Es wird die schlußendliche
Versendung der Nachrichten via CAN erläutert. Weiterhin wird ein Ausblick
gegeben, wie das Projekt weiter geführt werden kann und welche Punkte im Laufe
der Arbeit auftauchten, die verbessert werden können.


\minisec{\label{abstract}Abstract}

This paper is about designing a real time algorithm recognizing objects with the
aid of artificial neural networks in matlab. The algorithm is installed on a
Jetson Nano from Nvidia and executed there on the graphics processor unit (GPU).
If objects are recognized, the information is passed to a processing pipeline.
The pipeline manipulates the data obtained as desired. At the end, the
information about the objects should be sent via the Control Area Network (CAN).
Since the Jetson Nano does not have its own CAN controller, a possibility is
discussed how communication can be implemented via the bus.\\

The first part discusses what an artificial neural network is and why it is
suitable for object recognition. In the second part the Jetson Nano is
presented, that is to say the hardware (development board, circuit board). The
third part introduces the creation and installation of the object recognition
algorithm. In the fourth part, the processing pipeline is discussed, what it is
and how it fits into the project. In the fifth and last part, the individual
steps of the work will be considered again in a larger context. The final
transmission of the messages via CAN is explained. Furthermore, an outlook is
given of how the project can be continued and which points emerged in the course
of the work that can be improved.
